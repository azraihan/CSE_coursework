\section{Introduction}



\begin{frame}{Background}
    \begin{columns}
        % Column 1: Text
        \column{0.6\textwidth}
        \textbf{\textit{Generative AI tools like ChatGPT are revolutionizing various domains, including software engineering. But can we trust their responses?}}
        
        % Column 2: Image
        \column{0.4\textwidth}
        \includegraphics[width=\textwidth]{image2.jpg}
    \end{columns}
\end{frame}

%------------------------------------------------
\begin{frame}
    \frametitle{Background}
    \vspace{0.5cm}
    \textit{Developers are increasingly using ChatGPT for SE tasks like library selection}
    \vspace{0.5cm}
    \begin{figure}[!ht]
        \centering
        \resizebox{0.6\textwidth}{!}{% Adjusted the size to make it bigger
        \begin{circuitikz}
        \tikzstyle{every node}=[font=\Large]        
        \draw (10.75,11.5) rectangle (10.75,11.5);
        \draw [fill={rgb,255:red,209; green,240; blue,255}, rounded corners=7.2] (9.75,12.25) rectangle (14.5,11);
        \draw [fill={rgb,255:red,209; green,240; blue,255}, rounded corners=7.2] (9.5,9.5) rectangle (14.5,8.25);
        \draw [fill={rgb,255:red,209; green,240; blue,255}, rounded corners=6.0] (6,6.25) rectangle (9,5);
        \draw [fill={rgb,255:red,209; green,240; blue,255}, rounded corners=6.0] (10.75,6.25) rectangle (13.5,5);
        \draw [fill={rgb,255:red,209; green,240; blue,255}, rounded corners=6.0] (15.25,6.25) rectangle (18,5);
        \draw [short] (7.5,6.25) -- (7.5,7.25);
        \draw [short] (12,6.25) -- (12,7.25);
        \draw [short] (8.25,7.25) -- (16.75,7.25);
        \draw [short] (16.75,7.25) -- (16.75,6.25);
        \draw [short] (7.5,7.25) -- (8.25,7.25);
        \node at (12.15,11.6) {Driver Code};
        \node at (12,8.9) {Framework};
        \node at (7.5,5.5) {Library};
        \node at (12.15,5.5) {Library};
        \node at (16.65,5.5) {Library};
        \node [font=\small] at (12.25,10.25) {Code uses framework};
        \node [font=\small] at (12,7.5) {Framework contains library};
        \draw [short] (12,9.5) -- (12,10);
        \draw [->, >=Stealth] (12,10.5) -- (12,11);
        \draw [->, >=Stealth] (12,8) -- (12,8.25);
        \draw [short] (12,8) -- (12,7.75);
        \end{circuitikz}
        }%
        \label{meumeu}
    \end{figure}
\end{frame}
%------------------------------------------------
\begin{frame}
    \frametitle{Overview}
    \begin{figure}[!ht]
        \centering
        \resizebox{0.55\textwidth}{!}{%
        \begin{circuitikz}
        \tikzstyle{every node}=[font=\LARGE]
        \node [font=\LARGE] at (2.75,10.25) {};
        \draw (6,11.75) to[short] (6,4.25);
        \draw [->, >=Stealth] (6,4.25) -- (8.25,4.25);
        \node [font=\Large] at (10.75,12.75) {\underline{Phase 1}};
        \node [font=\large] at (10.75,12) {Survey to learn };
        \node [font=\large] at (10.75,11.5) {preferences and };
        \draw  (10.75,12) circle (2.25cm);
        \node [font=\large, rotate around={-360:(0,0)}] at (10.75,11) {concerns};
        \draw [->, >=Stealth] (6,11.75) -- (8.5,11.75);
        \draw [->, >=Stealth] (4.75,8) -- (6,8);
        \node [font=\Large] at (10.75,5.25) {\underline{Phase 2}};
        \node [font=\large] at (10.75,4.5) {Developed a tool};
        \node [font=\large] at (10.85,3.5) {ChatGPT Incorrectness};
        \node [font=\large] at (10.75,4) {called CID aka};
        \node [font=\large, rotate around={-360:(0,0)}] at (10.75,3) {Detector};
        \draw  (10.75,4.25) circle (2.5cm);
        \draw [ fill={rgb,255:red,204; green,245; blue,255} , rounded corners = 11.4, rotate around={-360:(2.75, 8)}] (0.75,8.75) rectangle (4.75,7.25);
        \node [font=\LARGE] at (2.75,8) {Research};
        \end{circuitikz}
        }%
        \label{meumeu}
        \end{figure}
\end{frame}

%------------------------------------------------
\setbeamercovered{transparent}
\begin{frame}
    \frametitle{Overview}
    \begin{itemize}
        \item Focus on software library selection as a case study. \pause
        \item Need for understanding how developers use ChatGPT and their concerns. \pause
        \item Desire for automated tools to detect incorrectness in ChatGPT's outputs.
    \end{itemize}
\end{frame}

%------------------------------------------------

\section{Survey of Software Developers}

%------------------------------------------------

\begin{frame}
    \frametitle{Survey Overview}
    \begin{itemize}
        \item Conducted a survey with 135 SE practitioners.
        \item Aimed to answer three Research Questions (RQs):
        \begin{itemize}
            \item \textbf{RQ1:} Why do software developers use ChatGPT?
            \item \textbf{RQ2:} How much do developers rely on ChatGPT responses?
            \item \textbf{RQ3:} How do developers verify ChatGPT responses?
        \end{itemize}
    \end{itemize}
\end{frame}

%-----------------------------

\begin{frame}
    \frametitle{Participant Demographics (Current Profession)}
    \vspace{0.5cm}
    \begin{tikzpicture}
        \pie[
            text=legend,
            radius=2.5,
            color={red!40, orange!40, green!40, yellow!40},
            before number=,
            after number=,
            style=drop shadow,
            sum=135
        ]{
            113/Software Engineer (Total 113),
            15/Manager (Total 15),
            4/Executive (Total 4),
            3/Non-Tech (Total 3)
        }
        \end{tikzpicture}
        
        \vspace{1em} % Add some vertical spacing
        \begin{center}
            \textbf{Total Participants = 135}
        \end{center}
\end{frame}


%------------------------------------------------

\begin{frame}
    \frametitle{Participant Demographics (Years of Experience)}
    \vspace{0.5cm}
    \begin{tikzpicture}
        \pie[
            text=legend,
            radius=2.5,
            color={blue!40, red!40, orange!40, green!40, yellow!40},
            before number=,
            after number=,
            style=drop shadow,
            sum=135
        ]{
            64/0-5 years,
            39/5-10 years,
            23/11-15 years,
            7/16-20 years,
            2/21-24 years
        }
        \end{tikzpicture}
        
        \vspace{1em} % Add some vertical spacing
        \begin{center}
            \textbf{Total Participants = 135}
        \end{center}
\end{frame}



%--------------------------------------------

\begin{frame}
    \frametitle{Survey Questions}
    {\scriptsize
        \begin{table}
\caption{Survey questions and their mapping to
the Research Questions. Here, C/O=Close/Open-ended question, G/S=Generic/Scenario-based question. For scenario-based questions, we used library selection as a case-study.}
\label{tab:survey_questions_on_chatgpt}
\resizebox{\linewidth}{!}{
\begin{tabular}{c p{9cm} c c c}
\hline
\textbf{Q\#} & \textbf{Questions} & \textbf{O/C} & \textbf{G/S} & \textbf{RQ} \\ \hline
1 & Did you use ChatGPT? & C & G & 1.1 \\
2 & In general, which of the cases you used it for? & C & G & 1.1 \\
3 & As a software professional, how did you or can you use it? & C & G & 1.1 \\
4 & How would you describe your experience with using it so far? & C & G & 1.1 \\
5 & How much do you rely on the content/response of ChatGPT? & C & G & 1.2 \\
6 & Have you considered using ChatGPT to select or compare software libraries? 
    Please share the pros and cons. & O & S & 1.2 \\
7 & How much would you rely on ChatGPT's response for the given library selection query? & C & S & 1.3 \\
8 & Would you rely on the ChatGPT's response after further inquiry? & C & S & 1.3 \\
9 & Do you think the opinion from ChatGPT is correct? & C & S & 1.3 \\
10 & What can be the ways to improve the reliability of ChatGPT responses? & C & G & 1.3 \\ \hline
\end{tabular}
}
\vspace{-4mm}
\end{table}

    }
\end{frame}


%------------------------------------------------
\begin{frame}
    \frametitle{Reasons to use ChatGPT (RQ1)}
    \textbf{The answers to the following questions help us to find the answer to Why developers used ChatGPT}
    \vspace{1cm}
    \begin{enumerate}
        \item Did you use ChatGPT?
        \item In general, which of the cases you used it for?
        \item As a software professional, how did you or can you use it? 
        \item How would you describe your experience with using it so far? 
    \end{enumerate}

\end{frame}

%--------------------------
\begin{frame}
    \frametitle{Reasons to use ChatGPT (RQ1)}
    \vspace{2em}
    \textbf{1. Did you use ChatGPT}
    \vspace{1cm}
    \begin{tikzpicture}
        \begin{axis}[
        xbar,
        xmin=0,
        xmax=100,
        width=0.8\linewidth,
        height=0.35\textheight,
        enlarge y limits=0.15,
        symbolic y coords={
        YES,NO
        },
        ytick=data,
            y tick label style={align=left, font=\footnotesize}, % Correct alignment
            nodes near coords,
            nodes near coords align={horizontal},
            axis lines*=left,
            y axis line style={opacity=0},
            x axis line style={opacity=0},
            tickwidth=0pt,
            enlarge x limits=0.02,
        ]
        \addplot[fill=red!80!white] coordinates {
        (98.52,YES)
        (1.48,NO)
        };
        \end{axis}
        \end{tikzpicture}
\end{frame}


%--------------------------

\begin{frame}
    \frametitle{Reasons to use ChatGPT (RQ1)}
    \vspace{1em}
    \textbf{2. In general, which of the cases you used it for?}
    \begin{tikzpicture}
        \begin{axis}[
            xbar,
            xmin=0,
            xmax=100,
            width=\linewidth,
            height=\textheight,
            bar width=6pt,
            enlarge y limits=0.15,
            symbolic y coords={
                Others,
                {Research for \\ Product Dev.},
                {Data Collection},
                Decision Making,
                Content Creation,
                {Learning \& \\ Knowledge Acq.},
                As a Search Engine,
                Just for Fun
            },
            ytick=data,
            y tick label style={align=left, font=\footnotesize},
            nodes near coords,
            nodes near coords align={horizontal},
            axis lines*=left,
            y axis line style={opacity=0},
            x axis line style={opacity=0},
            tickwidth=0pt,
            enlarge x limits=0.02,
        ]
        % Bars appear incrementally
        
        \only<1>{\addplot[fill=blue!80!white] coordinates {(42.22,Just for Fun)};}
        \only<2>{\addplot[fill=blue!80!white] coordinates {(42.22,Just for Fun) (79.26,As a Search Engine)};}
        \only<3>{\addplot[fill=blue!80!white] coordinates {(42.22,Just for Fun) (79.26,As a Search Engine) (61.48,{Learning \& \\ Knowledge Acq.})};}
        \only<4>{\addplot[fill=blue!80!white] coordinates {(42.22,Just for Fun) (79.26,As a Search Engine) (61.48,{Learning \& \\ Knowledge Acq.}) (46.67,Content Creation)};}
        \only<5>{\addplot[fill=blue!80!white] coordinates {(42.22,Just for Fun) (79.26,As a Search Engine) (61.48,{Learning \& \\ Knowledge Acq.}) (46.67,Content Creation) (48.15,Decision Making)};}
        \only<6>{\addplot[fill=blue!80!white] coordinates {(42.22,Just for Fun) (79.26,As a Search Engine) (61.48,{Learning \& \\ Knowledge Acq.}) (46.67,Content Creation) (48.15,Decision Making) (27.41,{Data Collection})};}
        \only<7>{\addplot[fill=blue!80!white] coordinates {(42.22,Just for Fun) (79.26,As a Search Engine) (61.48,{Learning \& \\ Knowledge Acq.}) (46.67,Content Creation) (48.15,Decision Making) (27.41,{Data Collection}) (28.15,{Research for \\ Product Dev.})};}
        \only<8>{\addplot[fill=blue!80!white] coordinates {(42.22,Just for Fun) (79.26,As a Search Engine) (61.48,{Learning \& \\ Knowledge Acq.}) (46.67,Content Creation) (48.15,Decision Making) (48.15,Decision Making) (27.41,{Data Collection}) (28.15,{Research for \\ Product Dev.}) (1.32,Others)};}
        \end{axis}
    \end{tikzpicture}
\end{frame}


%------------------------


\begin{frame}
    \frametitle{Reasons to use ChatGPT (RQ1)}
    \vspace{1em}
    \textbf{3.  As a software professional, how did you or can you use it?}
    \begin{tikzpicture}
        \begin{axis}[
            xbar,
            xmin=0,
            xmax=100,
            width=\linewidth,
            height=\textheight,
            bar width=6pt,
            enlarge y limits=0.15,
            symbolic y coords={
                {Code Generation \\ and Optimization},
                {Code Analysis \\ and review},
                {Solving \\ Support},
                {Alternative \\ Approaches},
                {Library \\ Selection},Others
            },
            ytick=data,
            y tick label style={align=left, font=\footnotesize},
            nodes near coords,
            nodes near coords align={horizontal},
            axis lines*=left,
            y axis line style={opacity=0},
            x axis line style={opacity=0},
            tickwidth=0pt,
            enlarge x limits=0.02,
        ]
        % Bars appear incrementally
        
        \only<1>{\addplot[fill=green!80!white] coordinates {(60,{Code Generation \\ and Optimization})};}
        \only<2>{\addplot[fill=green!80!white] coordinates {(60,{Code Generation \\ and Optimization}) (52.59,{Code Analysis \\ and review})};}
        \only<3>{\addplot[fill=green!80!white] coordinates {(60,{Code Generation \\ and Optimization}) (52.59,{Code Analysis \\ and review}) (86,{Solving \\ Support})};}
        \only<4>{\addplot[fill=green!80!white] coordinates {(60,{Code Generation \\ and Optimization}) (52.59,{Code Analysis \\ and review}) (86,{Solving \\ Support}) (73.33,{Alternative \\ Approaches})};}
        \only<5>{\addplot[fill=green!80!white] coordinates {(60,{Code Generation \\ and Optimization}) (52.59,{Code Analysis \\ and review}) (86,{Solving \\ Support}) (73.33,{Alternative \\ Approaches}) (46.67,{Library \\ Selection})};}
        \only<6>{\addplot[fill=green!80!white] coordinates {(60,{Code Generation \\ and Optimization}) (52.59,{Code Analysis \\ and review}) (86,{Solving \\ Support}) (73.33,{Alternative \\ Approaches}) (46.67,{Library \\ Selection}) (2,Others)};}
        \end{axis}
    \end{tikzpicture}
\end{frame}

%---------------------------

\begin{frame}
    \frametitle{Reasons to use ChatGPT (RQ1)}
    \vspace{1em}
    \textbf{4. How would you describe your experience with using it so far?}
    \begin{tikzpicture}
        \begin{axis}[
            xbar,
            xmin=0,
            xmax=100,
            width=\linewidth,
            height=0.7\textheight,
            bar width=6pt,
            enlarge y limits=0.15,
            symbolic y coords={
                Exciting,
                {Tech \\ Marvel},
                Unreliable,
                Over-hyped,
                Suspicious
            },
            ytick=data,
            y tick label style={align=left, font=\footnotesize},
            nodes near coords,
            nodes near coords align={horizontal},
            axis lines*=left,
            y axis line style={opacity=0},
            x axis line style={opacity=0},
            tickwidth=0pt,
            enlarge x limits=0.02,
        ]
        \addplot[fill=purple!80!white] coordinates {
            (68.15,Exciting)
            (42.22,{Tech \\ Marvel})
            (34,Unreliable)
            (19.26,Over-hyped)
            (9.63,Suspicious)
        };
        \end{axis}
    \end{tikzpicture}
\end{frame}

%----------------------------

\begin{frame}
    \frametitle{Summarizing Key Findings for RQ1}
    \begin{itemize}
        \item \textbf{Usage Purposes:}
        \begin{itemize}
            \item Code generation and optimization.
            \item Problem-solving support.
            \item Exploring alternative approaches.
            \item Library selection.
        \end{itemize}
        \item \textbf{Experience:}
        \begin{itemize}
            \item Excitement and recognition of technological advancement.
            \item Concerns about reliability and overhyped expectations.
        \end{itemize}
    \end{itemize}
\end{frame}



%------------------------------------------------


\begin{frame}
    \frametitle{Concerns About ChatGPT Responses (RQ2)}
    \textbf{We asked the following questions to the participants to find out how reliable ChatGPT is}
    \vspace{1cm}
    \begin{enumerate}
        \item How much do you rely on the content/response of ChatGPT?
        \item Have you considered using ChatGPT to select or compare software
        libraries? Please share the pros and cons?
    \end{enumerate}

\end{frame}

%----------------------------------------------
\begin{frame}
    \frametitle{Concerns About ChatGPT Responses (RQ2)}
    \textbf{1. How much do you rely on the content/response of ChatGPT?}
    \vspace{1cm}
    \begin{tikzpicture}
        \pie[
            text=legend,
            radius=1.9,
            color={blue!40, red!40, orange!40, green!40, yellow!40},
            before number=,
            after number={\%},
            style=drop shadow,
            explode=0.07
        ]{
            5.2/Not Reliable At All,
            54.9/Somewhat Reliable:Need Validation,
            35.6/Somewhat Reliable:Need Augmentation,
            3.7/Others
        }
        \end{tikzpicture}
\end{frame}

%------------------------------------------------

\begin{frame}
    \frametitle{Concerns About ChatGPT Responses (RQ2)}
    \vspace{1em}
    \textbf{2. Have you considered using ChatGPT to select or compare
    software libraries? Please share the pros and cons?}
    \vspace{0.5em}
    \begin{itemize}
        \item PROS :
        \begin{enumerate}
            \item Efficient Access to Information
            \item Initial Idea Generation
            \item Personalized Recommendations
            \item Time-Saving
        \end{enumerate}
        \vspace{0.5em}
        \item CONS :
        \begin{enumerate}
            \item Lack of Up-to-dateness
            \item Contextual Understanding Challenges
            \item Reliability Concerns
            \item Dependence on Prompt
            \item Not Sufficient for Decision-Making
            \item Bias of Training Data
        \end{enumerate}
        
    \end{itemize}
    
\end{frame}


%--------------------------------
\begin{frame}
    \frametitle{Pros and Cons for Library Selection (RQ2)}
    \vspace{0.5em}

    \begin{minipage}{\textwidth}
        \textbf{Pros : Total 45 responses}
        \centering
        \begin{tikzpicture}
            \pie[text=legend, radius=1.25, sum=45, style=drop shadow]{
                22/Efficient Access to Info,
                10/Initial Idea Generation,
                3/Personalized Recommendations,
                10/Time Saving
            }
        \end{tikzpicture}
    \end{minipage}

    \vspace{1em} % Adds space between the pie charts

    \begin{minipage}{\textwidth}
        \textbf{Cons : Total 60 Responses}
        \centering
        \begin{tikzpicture}
            \pie[text=legend, radius=1.25, color={red!70!white, orange!70!white, gray!50, purple!50, green!50, brown!50}, sum=62, style=drop shadow]{
                12/Lack of Up-to-date Knowledge,
                26/Reliability Concerns,
                11/Contextual Understanding Challenges,
                4/Dependence on Prompt,
                7/Not Sufficient for Decision Making,
                2/Biased on Training Data
            }
        \end{tikzpicture}
    \end{minipage}

\end{frame}

%-------------------------------------------------
\begin{frame}
    \frametitle{Key Findings for RQ2}
    \begin{itemize}
        \item \textbf{Reliability Concerns:}
        \begin{itemize}
            \item Only a small percentage fully trust ChatGPT responses.
            \item Majority consider the responses somewhat reliable but require validation.
        \end{itemize}
    \end{itemize}

      % Define colors within the frame
  \begin{tikzpicture}[scale=0.5, transform shape]
    % Local color definitions
    \definecolor{DeciderColor}{RGB}{102, 102, 238}    % Blue
    \definecolor{ChallengerColor}{RGB}{119, 209, 243}  % Green
    \definecolor{EnqColor}{RGB}{241, 230, 145}        % Orange
    \definecolor{MixedColor}{RGB}{244, 182, 114}     % Purple
    \definecolor{GreyColor}{RGB}{192, 192, 192}     % Grey for blurred slices

    % Overlay 1: Highlight Decider
    \only<1>{
        \pie[
      cloud,
      text=legend,
      scale font,
      radius=3, % Increased radius for more space
      % Optionally adjust font size
      color={
          DeciderColor,
          ChallengerColor,
          EnqColor,
          MixedColor
        },
      font=\scriptsize
    ]{
      34/Unreliaable,
      19.26/Over hyped,
      2.74/Fully Reliable, % Abbreviated label
      54/Somewhat Reliable
    }
    }
    \only<2>{
      \pie[
        cloud,
        text=legend,
        scale font,
        radius=3,
        color={
          DeciderColor,
          GreyColor,
          GreyColor,
          GreyColor
        },
        font=\scriptsize
      ]{
        34/Unreliaable,
        19.26/Over hyped,
        2.74/Fully Reliable, % Abbreviated label
        54/Somewhat Reliable
      }
    }

    % Overlay 2: Highlight Challenger
    \only<3>{
      \pie[
        cloud,
        text=legend,
        scale font,
        radius=3,
        color={
          GreyColor,
          ChallengerColor,
          GreyColor,
          GreyColor
        },
        font=\scriptsize
      ]{
        34/Unreliaable,
        19.26/Over hyped,
        2.74/Fully Reliable, % Abbreviated label
        54/Somewhat Reliable
      }
    }

    % Overlay 3: Highlight Enquirer
    \only<4>{
      \pie[
        cloud,
        text=legend,
        scale font,
        radius=3,
        color={
          GreyColor,
          GreyColor,
          EnqColor,
          GreyColor
        },
        font=\scriptsize
      ]{
        34/Unreliaable,
        19.26/Over hyped,
        2.74/Fully Reliable, % Abbreviated label
        54/Somewhat Reliable
      }
    }

    % Overlay 4: Highlight Mixed
    \only<5>{
      \pie[
        cloud,
        text=legend,
        scale font,
        radius=3,
        color={
          GreyColor,
          GreyColor,
          GreyColor,
          MixedColor
        },
        font=\scriptsize
      ]{
        34/Unreliaable,
        19.26/Over hyped,
        2.74/Fully Reliable, % Abbreviated label
        54/Somewhat Reliable
      }
    }
  \end{tikzpicture}
\end{frame}

%------------------------------------------------
